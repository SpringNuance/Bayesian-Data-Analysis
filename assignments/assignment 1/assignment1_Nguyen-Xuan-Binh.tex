% Options for packages loaded elsewhere
\PassOptionsToPackage{unicode}{hyperref}
\PassOptionsToPackage{hyphens}{url}
\PassOptionsToPackage{dvipsnames,svgnames,x11names}{xcolor}
%
\documentclass[
]{article}
\title{BDA - Assignment 1}
\author{Anonymous}
\date{}

\usepackage{amsmath,amssymb}
\usepackage{lmodern}
\usepackage{iftex}
\ifPDFTeX
  \usepackage[T1]{fontenc}
  \usepackage[utf8]{inputenc}
  \usepackage{textcomp} % provide euro and other symbols
\else % if luatex or xetex
  \usepackage{unicode-math}
  \defaultfontfeatures{Scale=MatchLowercase}
  \defaultfontfeatures[\rmfamily]{Ligatures=TeX,Scale=1}
\fi
% Use upquote if available, for straight quotes in verbatim environments
\IfFileExists{upquote.sty}{\usepackage{upquote}}{}
\IfFileExists{microtype.sty}{% use microtype if available
  \usepackage[]{microtype}
  \UseMicrotypeSet[protrusion]{basicmath} % disable protrusion for tt fonts
}{}
\makeatletter
\@ifundefined{KOMAClassName}{% if non-KOMA class
  \IfFileExists{parskip.sty}{%
    \usepackage{parskip}
  }{% else
    \setlength{\parindent}{0pt}
    \setlength{\parskip}{6pt plus 2pt minus 1pt}}
}{% if KOMA class
  \KOMAoptions{parskip=half}}
\makeatother
\usepackage{xcolor}
\IfFileExists{xurl.sty}{\usepackage{xurl}}{} % add URL line breaks if available
\IfFileExists{bookmark.sty}{\usepackage{bookmark}}{\usepackage{hyperref}}
\hypersetup{
  pdftitle={BDA - Assignment 1},
  pdfauthor={Anonymous},
  colorlinks=true,
  linkcolor={Maroon},
  filecolor={Maroon},
  citecolor={Blue},
  urlcolor={blue},
  pdfcreator={LaTeX via pandoc}}
\urlstyle{same} % disable monospaced font for URLs
\usepackage[margin=1in]{geometry}
\usepackage{color}
\usepackage{fancyvrb}
\newcommand{\VerbBar}{|}
\newcommand{\VERB}{\Verb[commandchars=\\\{\}]}
\DefineVerbatimEnvironment{Highlighting}{Verbatim}{commandchars=\\\{\}}
% Add ',fontsize=\small' for more characters per line
\usepackage{framed}
\definecolor{shadecolor}{RGB}{248,248,248}
\newenvironment{Shaded}{\begin{snugshade}}{\end{snugshade}}
\newcommand{\AlertTok}[1]{\textcolor[rgb]{0.94,0.16,0.16}{#1}}
\newcommand{\AnnotationTok}[1]{\textcolor[rgb]{0.56,0.35,0.01}{\textbf{\textit{#1}}}}
\newcommand{\AttributeTok}[1]{\textcolor[rgb]{0.77,0.63,0.00}{#1}}
\newcommand{\BaseNTok}[1]{\textcolor[rgb]{0.00,0.00,0.81}{#1}}
\newcommand{\BuiltInTok}[1]{#1}
\newcommand{\CharTok}[1]{\textcolor[rgb]{0.31,0.60,0.02}{#1}}
\newcommand{\CommentTok}[1]{\textcolor[rgb]{0.56,0.35,0.01}{\textit{#1}}}
\newcommand{\CommentVarTok}[1]{\textcolor[rgb]{0.56,0.35,0.01}{\textbf{\textit{#1}}}}
\newcommand{\ConstantTok}[1]{\textcolor[rgb]{0.00,0.00,0.00}{#1}}
\newcommand{\ControlFlowTok}[1]{\textcolor[rgb]{0.13,0.29,0.53}{\textbf{#1}}}
\newcommand{\DataTypeTok}[1]{\textcolor[rgb]{0.13,0.29,0.53}{#1}}
\newcommand{\DecValTok}[1]{\textcolor[rgb]{0.00,0.00,0.81}{#1}}
\newcommand{\DocumentationTok}[1]{\textcolor[rgb]{0.56,0.35,0.01}{\textbf{\textit{#1}}}}
\newcommand{\ErrorTok}[1]{\textcolor[rgb]{0.64,0.00,0.00}{\textbf{#1}}}
\newcommand{\ExtensionTok}[1]{#1}
\newcommand{\FloatTok}[1]{\textcolor[rgb]{0.00,0.00,0.81}{#1}}
\newcommand{\FunctionTok}[1]{\textcolor[rgb]{0.00,0.00,0.00}{#1}}
\newcommand{\ImportTok}[1]{#1}
\newcommand{\InformationTok}[1]{\textcolor[rgb]{0.56,0.35,0.01}{\textbf{\textit{#1}}}}
\newcommand{\KeywordTok}[1]{\textcolor[rgb]{0.13,0.29,0.53}{\textbf{#1}}}
\newcommand{\NormalTok}[1]{#1}
\newcommand{\OperatorTok}[1]{\textcolor[rgb]{0.81,0.36,0.00}{\textbf{#1}}}
\newcommand{\OtherTok}[1]{\textcolor[rgb]{0.56,0.35,0.01}{#1}}
\newcommand{\PreprocessorTok}[1]{\textcolor[rgb]{0.56,0.35,0.01}{\textit{#1}}}
\newcommand{\RegionMarkerTok}[1]{#1}
\newcommand{\SpecialCharTok}[1]{\textcolor[rgb]{0.00,0.00,0.00}{#1}}
\newcommand{\SpecialStringTok}[1]{\textcolor[rgb]{0.31,0.60,0.02}{#1}}
\newcommand{\StringTok}[1]{\textcolor[rgb]{0.31,0.60,0.02}{#1}}
\newcommand{\VariableTok}[1]{\textcolor[rgb]{0.00,0.00,0.00}{#1}}
\newcommand{\VerbatimStringTok}[1]{\textcolor[rgb]{0.31,0.60,0.02}{#1}}
\newcommand{\WarningTok}[1]{\textcolor[rgb]{0.56,0.35,0.01}{\textbf{\textit{#1}}}}
\usepackage{graphicx}
\makeatletter
\def\maxwidth{\ifdim\Gin@nat@width>\linewidth\linewidth\else\Gin@nat@width\fi}
\def\maxheight{\ifdim\Gin@nat@height>\textheight\textheight\else\Gin@nat@height\fi}
\makeatother
% Scale images if necessary, so that they will not overflow the page
% margins by default, and it is still possible to overwrite the defaults
% using explicit options in \includegraphics[width, height, ...]{}
\setkeys{Gin}{width=\maxwidth,height=\maxheight,keepaspectratio}
% Set default figure placement to htbp
\makeatletter
\def\fps@figure{htbp}
\makeatother
\setlength{\emergencystretch}{3em} % prevent overfull lines
\providecommand{\tightlist}{%
  \setlength{\itemsep}{0pt}\setlength{\parskip}{0pt}}
\setcounter{secnumdepth}{-\maxdimen} % remove section numbering
\ifLuaTeX
  \usepackage{selnolig}  % disable illegal ligatures
\fi

\begin{document}
\maketitle

{
\hypersetup{linkcolor=}
\setcounter{tocdepth}{1}
\tableofcontents
}
\hypertarget{loaded-packages}{%
\section{Loaded packages}\label{loaded-packages}}

\begin{Shaded}
\begin{Highlighting}[]
\FunctionTok{library}\NormalTok{(aaltobda)}
\FunctionTok{library}\NormalTok{(ggplot2)}
\FunctionTok{library}\NormalTok{(rstan)}
\end{Highlighting}
\end{Shaded}

\begin{verbatim}
## Warning: package 'rstan' was built under R version 4.1.3
\end{verbatim}

\begin{verbatim}
## Loading required package: StanHeaders
\end{verbatim}

\begin{verbatim}
## Warning: package 'StanHeaders' was built under R version 4.1.3
\end{verbatim}

\begin{verbatim}
## 
## rstan version 2.26.13 (Stan version 2.26.1)
\end{verbatim}

\begin{verbatim}
## For execution on a local, multicore CPU with excess RAM we recommend calling
## options(mc.cores = parallel::detectCores()).
## To avoid recompilation of unchanged Stan programs, we recommend calling
## rstan_options(auto_write = TRUE)
## For within-chain threading using `reduce_sum()` or `map_rect()` Stan functions,
## change `threads_per_chain` option:
## rstan_options(threads_per_chain = 1)
\end{verbatim}

\begin{verbatim}
## Do not specify '-march=native' in 'LOCAL_CPPFLAGS' or a Makevars file
\end{verbatim}

\begin{Shaded}
\begin{Highlighting}[]
\FunctionTok{library}\NormalTok{(markmyassignment)}
\end{Highlighting}
\end{Shaded}

\begin{verbatim}
## Warning: package 'markmyassignment' was built under R version 4.1.3
\end{verbatim}

\begin{Shaded}
\begin{Highlighting}[]
\NormalTok{assignment\_path }\OtherTok{\textless{}{-}}
\FunctionTok{paste}\NormalTok{(}\StringTok{"https://github.com/avehtari/BDA\_course\_Aalto/"}\NormalTok{,}
\StringTok{"blob/master/assignments/tests/assignment1.yml"}\NormalTok{, }\AttributeTok{sep=}\StringTok{""}\NormalTok{)}
\FunctionTok{set\_assignment}\NormalTok{(assignment\_path)}
\end{Highlighting}
\end{Shaded}

\begin{verbatim}
## Assignment set:
## assignment1: Bayesian Data Analysis: Assignment 1
## The assignment contain the following (3) tasks:
## - p_red
## - p_box
## - p_identical_twin
\end{verbatim}

\hypertarget{exercise-1-basic-probability-theory-notation-and-terms}{%
\section{Exercise 1 (Basic probability theory notation and
terms)}\label{exercise-1-basic-probability-theory-notation-and-terms}}

Explain each of the following terms with one sentence: - probability:
Probability is a way of quantifying the belief of something being true
or false.\\
- probability mass\\
- probability density\\
- probability mass function (pmf)\\
- probability density function (pdf)\\
- probability distribution\\
- discrete probability distribution\\
- continuous probability distribution\\
- cumulative distribution function (cdf)\\
- likelihood\\

\hypertarget{exercise-2-basic-computer-skills}{%
\section{Exercise 2 (Basic computer
skills)}\label{exercise-2-basic-computer-skills}}

\hypertarget{a}{%
\subsection{a)}\label{a}}

Plot the density function of Beta-distribution, with mean = 0.2 and
variance = 0.01. The parameters α and β of the Beta-distribution are
related to the mean and variance Hint! Useful R functions: seq(), plot()
and dbeta() define range

\begin{Shaded}
\begin{Highlighting}[]
\NormalTok{p }\OtherTok{=} \FunctionTok{seq}\NormalTok{(}\DecValTok{0}\NormalTok{, }\DecValTok{1}\NormalTok{, }\AttributeTok{length=}\DecValTok{100}\NormalTok{)}
\end{Highlighting}
\end{Shaded}

create plot of Beta distribution with mean = 0.2 and variance = 0.01

\begin{Shaded}
\begin{Highlighting}[]
\NormalTok{mu }\OtherTok{=} \FloatTok{0.2}
\NormalTok{sigma2 }\OtherTok{=} \FloatTok{0.01}
\NormalTok{alpha }\OtherTok{=}\NormalTok{ mu }\SpecialCharTok{*}\NormalTok{ ( ((mu }\SpecialCharTok{*}\NormalTok{ (}\DecValTok{1} \SpecialCharTok{{-}}\NormalTok{ mu))}\SpecialCharTok{/}\NormalTok{sigma2) }\SpecialCharTok{{-}} \DecValTok{1}\NormalTok{)}
\NormalTok{beta }\OtherTok{=}\NormalTok{ (alpha }\SpecialCharTok{*}\NormalTok{ (}\DecValTok{1} \SpecialCharTok{{-}}\NormalTok{ mu))}\SpecialCharTok{/}\NormalTok{mu}
\FunctionTok{plot}\NormalTok{(p, }\FunctionTok{dbeta}\NormalTok{(p, alpha, beta), }\AttributeTok{ylab=}\StringTok{\textquotesingle{}density\textquotesingle{}}\NormalTok{,}
     \AttributeTok{type =}\StringTok{\textquotesingle{}l\textquotesingle{}}\NormalTok{, }\AttributeTok{col=}\StringTok{\textquotesingle{}purple\textquotesingle{}}\NormalTok{, }\AttributeTok{main=}\StringTok{\textquotesingle{}Beta Distribution\textquotesingle{}}\NormalTok{)}
\end{Highlighting}
\end{Shaded}

\includegraphics{assignment1_Nguyen-Xuan-Binh_files/figure-latex/unnamed-chunk-3-1.pdf}
\#\# b) Take a sample of 1000 random numbers from the above distribution
and plot a histogram of the results. Compare visually to the density
function. Hint! Useful R functions: rbeta() and hist()

\begin{Shaded}
\begin{Highlighting}[]
\NormalTok{sample }\OtherTok{=} \FunctionTok{rbeta}\NormalTok{(}\DecValTok{1000}\NormalTok{, alpha, beta)}
\FunctionTok{hist}\NormalTok{(sample, }\AttributeTok{breaks=}\DecValTok{50}\NormalTok{)}
\end{Highlighting}
\end{Shaded}

\includegraphics{assignment1_Nguyen-Xuan-Binh_files/figure-latex/unnamed-chunk-4-1.pdf}
The histogram of random sampled points closely resemble the density
function of Beta distribution above

\hypertarget{c}{%
\subsection{c)}\label{c}}

Compute the sample mean and variance from the drawn sample. Verify that
they match (roughly) to the true mean and variance of the distribution.
Hint! Useful R functions: mean() and var() The sample mean is:

\begin{Shaded}
\begin{Highlighting}[]
\FunctionTok{mean}\NormalTok{(sample)}
\end{Highlighting}
\end{Shaded}

\begin{verbatim}
## [1] 0.196327
\end{verbatim}

The sample mean is close to the true mean = 0.2

The sample variance is:

\begin{Shaded}
\begin{Highlighting}[]
\FunctionTok{var}\NormalTok{(sample)}
\end{Highlighting}
\end{Shaded}

\begin{verbatim}
## [1] 0.009492876
\end{verbatim}

The sample variance is close to the true variance = 0.01

\hypertarget{d}{%
\subsection{d)}\label{d}}

Estimate the central 95\% probability interval of the distribution from
the drawn samples. Hint! Useful R functions: quantile()

\begin{Shaded}
\begin{Highlighting}[]
\FunctionTok{quantile}\NormalTok{(sample, }\AttributeTok{probs=}\FunctionTok{c}\NormalTok{(}\FloatTok{0.025}\NormalTok{, }\FloatTok{0.975}\NormalTok{))}
\end{Highlighting}
\end{Shaded}

\begin{verbatim}
##       2.5%      97.5% 
## 0.04867702 0.41789317
\end{verbatim}

\hypertarget{exercise-3-bayes-theorem}{%
\section{Exercise 3 (Bayes' theorem)}\label{exercise-3-bayes-theorem}}

A group of researchers has designed a new inexpensive and painless test
for detecting lung cancer. The test is intended to be an initial
screening test for the population in general. A positive result
(presence of lung cancer) from the test would be followed up immediately
with medication, surgery or more extensive and expensive test. The
researchers know from their studies the following facts:\\
- Test gives a positive result in 98\% of the time when the test subject
has lung cancer.\\
- Test gives a negative result in 96\% of the time when the test subject
does not have lung cancer.\\
- In general population approximately one person in 1000 has lung
cancer.\\
The researchers are happy with these preliminary results (about 97\%
success rate), and wish to get the test to market as soon as possible.
How would you advise them? Base your answer on Bayes' rule
computations.\\
Hint : Relatively high false negative (cancer doesn't get detected) or
high false positive (unnecessarily administered medication) rates are
typically bad and undesirable in tests.\\
Hint : Here are some probability values that can help you figure out if
you copied the right conditional probabilities from the question.\\
- P(Test gives positive \textbar{} Subject does not have lung cancer) =
4\%\\
- P(Test gives positive and Subject has lung cancer) = 0.098\% this is
also referred to as the joint probability of test being positive and the
subject having lung cancer\\

Let + sign be positive test result and - sign be negative test result
From the knowledge, we know that:\\
- \(P(+|have-cancer) = 98\% => P(-|have-cancer) = 2\%\)\\
- \(P(-|no-cancer) = 96\% => P(+|no-cancer) = 4\%\)\\
- \(P(have-cancer) = 0.1\% => P(no-cancer) = 99.9\%\)\\
-
\(P(+|have-cancer or -|no-cancer) = \frac{P(+|have-cancer) + P(+|no-cancer)}{2} = \frac{(98\% + 96\%)}{2} = 97\%\)\\
From these identities, we can calculate many probability values:\\
- \(P(+) = \frac{P(+|have-cancer) + P(+|no-cancer)}{2} = 51\%\)\\
- \(P(-) = \frac{P(-|have-cancer) + P(-|no-cancer)}{2} = 49\%\)\\
The posteriors can be calculated as: -
\(P(have cancer|+) = \frac{P(+|have cancer)P(have cancer)}{P(+)} = \frac{98\% 0.1\%}{51\%} \approx 19.21\%\)\\
\# Exercise 4 (Bayes' theorem) We have three boxes, A, B, and C. There
are - 2 red balls and 5 white balls in the box A, - 4 red balls and 1
white ball in the box B, and - 1 red ball and 3 white balls in the box
C. Consider a random experiment in which one of the boxes is randomly
selected and from that box, one ball is randomly picked up. After
observing the color of the ball it is replaced in the box it came from.
Suppose also that on average box A is selected 40\% of the time and box
B 10\% of the time (i.e.~P(A) = 0.4). a) What is the probability of
picking a red ball? b) If a red ball was picked, from which box it most
probably came from? Implement two functions in R that computes the
probabilities.

\begin{Shaded}
\begin{Highlighting}[]
\NormalTok{p\_red }\OtherTok{\textless{}{-}} \ControlFlowTok{function}\NormalTok{(boxes)\{}
  \DecValTok{13}
\NormalTok{\}}
\NormalTok{p\_box }\OtherTok{\textless{}{-}} \ControlFlowTok{function}\NormalTok{(boxes)\{}
  \DecValTok{12}
\NormalTok{\}}
\CommentTok{\# To check your code/functions, just run}
\FunctionTok{mark\_my\_assignment}\NormalTok{()}
\end{Highlighting}
\end{Shaded}

\begin{verbatim}
## Warning: package 'testthat' was built under R version 4.1.3
\end{verbatim}

\begin{verbatim}
## v | F W S  OK | Context
## 
## / |         0 | task-1-subtask-1-tests                                          
## / |         0 | p_red()                                                         
## x | 2       2 | p_red()
## --------------------------------------------------------------------------------
## Failure (test-task-1-subtask-1-tests.R:16:3): p_red()
## p_red(boxes = boxes) not equivalent to 0.3928571.
## 1/1 mismatches
## [1] 13 - 0.393 == 12.6
## Error: Incorrect result for matrix(c(2,2,1,5,5,1), ncol = 2)
## 
## Failure (test-task-1-subtask-1-tests.R:21:3): p_red()
## p_red(boxes = boxes) not equivalent to 0.5.
## 1/1 mismatches
## [1] 13 - 0.5 == 12.5
## Error: Incorrect result for matrix(c(1,1,1,1,1,1), ncol = 2)
## --------------------------------------------------------------------------------
## 
## / |         0 | task-2-subtask-1-tests                                          
## / |         0 | p_box()                                                         
## x | 2       2 | p_box()
## --------------------------------------------------------------------------------
## Failure (test-task-2-subtask-1-tests.R:15:3): p_box()
## p_box(boxes = boxes) not equivalent to c(0.29090909, 0.07272727, 0.63636364).
## Lengths differ: 1 is not 3
## Error: Incorrect result for matrix(c(2,2,1,5,5,1), ncol = 2)
## 
## Failure (test-task-2-subtask-1-tests.R:19:3): p_box()
## p_box(boxes = boxes) not equivalent to c(0.4, 0.1, 0.5).
## Lengths differ: 1 is not 3
## Error: Incorrect result for matrix(c(1,1,1,1,1,1), ncol = 2)
## --------------------------------------------------------------------------------
## 
## / |         0 | task-3-subtask-1-tests                                          
## / |         0 | p_identical_twin()                                              
## - | 1       0 | p_identical_twin()                                              
## x | 2       0 | p_identical_twin() [0.1s]
## --------------------------------------------------------------------------------
## Failure (test-task-3-subtask-1-tests.R:8:3): p_identical_twin()
## exists("p_identical_twin") is not TRUE
## 
## `actual`:   FALSE
## `expected`: TRUE 
## Error: p_identical_twin() is missing
## 
## Error (test-task-3-subtask-1-tests.R:10:3): p_identical_twin()
## Error in `checkFunction(x, args, ordered, nargs, null.ok)`: object 'p_identical_twin' not found
## Backtrace:
##  1. checkmate::expect_function(...)
##       at test-task-3-subtask-1-tests.R:10:2
##  2. checkmate::checkFunction(x, args, ordered, nargs, null.ok)
## --------------------------------------------------------------------------------
## 
## == Results =====================================================================
## Duration: 0.2 s
## 
## [ FAIL 6 | WARN 0 | SKIP 0 | PASS 4 ]
\end{verbatim}

\hypertarget{exercise-5-bayes-theorem}{%
\section{Exercise 5 (Bayes' theorem)}\label{exercise-5-bayes-theorem}}

Assume that on average fraternal twins (two fertilized eggs and then
could be of different sex) occur once in 150 births and identical twins
(single egg divides into two separate embryos, so both have the same
sex) once in 400 births (Note! This is not the true value, see Exercise
1.6, page 28, in BDA3). American male singer-actor Elvis Presley (1935 -
1977) had a twin brother who died in birth. Assume that an equal number
of boys and girls are born on average. What is the probability that
Elvis was an identical twin? Show the steps how you derived the
equations to compute that probability. Implement this as a function in R
that computes the probability.

\begin{Shaded}
\begin{Highlighting}[]
\NormalTok{p\_identical\_twin }\OtherTok{\textless{}{-}} \ControlFlowTok{function}\NormalTok{(fraternal\_prob, identical\_prob)\{}
  \FunctionTok{return}\NormalTok{(}\DecValTok{14}\NormalTok{)}
\NormalTok{\}}

\FunctionTok{mark\_my\_assignment}\NormalTok{()}
\end{Highlighting}
\end{Shaded}

\begin{verbatim}
## v | F W S  OK | Context
## 
## / |         0 | task-1-subtask-1-tests                                          
## / |         0 | p_red()                                                         
## x | 2       2 | p_red()
## --------------------------------------------------------------------------------
## Failure (test-task-1-subtask-1-tests.R:16:3): p_red()
## p_red(boxes = boxes) not equivalent to 0.3928571.
## 1/1 mismatches
## [1] 13 - 0.393 == 12.6
## Error: Incorrect result for matrix(c(2,2,1,5,5,1), ncol = 2)
## 
## Failure (test-task-1-subtask-1-tests.R:21:3): p_red()
## p_red(boxes = boxes) not equivalent to 0.5.
## 1/1 mismatches
## [1] 13 - 0.5 == 12.5
## Error: Incorrect result for matrix(c(1,1,1,1,1,1), ncol = 2)
## --------------------------------------------------------------------------------
## 
## / |         0 | task-2-subtask-1-tests                                          
## / |         0 | p_box()                                                         
## x | 2       2 | p_box()
## --------------------------------------------------------------------------------
## Failure (test-task-2-subtask-1-tests.R:15:3): p_box()
## p_box(boxes = boxes) not equivalent to c(0.29090909, 0.07272727, 0.63636364).
## Lengths differ: 1 is not 3
## Error: Incorrect result for matrix(c(2,2,1,5,5,1), ncol = 2)
## 
## Failure (test-task-2-subtask-1-tests.R:19:3): p_box()
## p_box(boxes = boxes) not equivalent to c(0.4, 0.1, 0.5).
## Lengths differ: 1 is not 3
## Error: Incorrect result for matrix(c(1,1,1,1,1,1), ncol = 2)
## --------------------------------------------------------------------------------
## 
## / |         0 | task-3-subtask-1-tests                                          
## / |         0 | p_identical_twin()                                              
## x | 3       2 | p_identical_twin()
## --------------------------------------------------------------------------------
## Failure (test-task-3-subtask-1-tests.R:13:3): p_identical_twin()
## p_identical_twin(fraternal_prob = (1/125), identical_prob = (1/300)) not equivalent to 0.4545455.
## 1/1 mismatches
## [1] 14 - 0.455 == 13.5
## Error: Incorrect result for fraternal_prob = 1/125 and identical_prob = 1/300
## 
## Failure (test-task-3-subtask-1-tests.R:16:3): p_identical_twin()
## p_identical_twin(fraternal_prob = 1/100, identical_prob = 1/500) not equivalent to 0.2857143.
## 1/1 mismatches
## [1] 14 - 0.286 == 13.7
## Error: Incorrect result for fraternal_prob = 1/100 and identical_prob = 1/500
## 
## Failure (test-task-3-subtask-1-tests.R:19:3): p_identical_twin()
## p_identical_twin(fraternal_prob = 1/10, identical_prob = 1/20) not equivalent to 0.5.
## 1/1 mismatches
## [1] 14 - 0.5 == 13.5
## Error: Incorrect result for fraternal_prob = 1/10 and identical_prob = 1/20
## --------------------------------------------------------------------------------
## 
## == Results =====================================================================
## Duration: 0.1 s
## 
## [ FAIL 7 | WARN 0 | SKIP 0 | PASS 6 ]
\end{verbatim}

\end{document}
